\documentclass[12pt]{article}
% \usepackage[utf8]{vietnam}
 \usepackage[latin1]{inputenc}  
\usepackage{amssymb,amsmath,amsthm,latexsym,enumerate}
\usepackage{graphicx,subfigure,psfrag} 
%\usepackage{url}
\usepackage{cite}
 \usepackage{longtable}
\usepackage{multirow}
\usepackage{color}
\newcommand{\red}[1]{{\color{red} #1 }}
\newcommand{\blue}[1]{{\color{blue} #1 }}

\addtolength{\voffset}{-2cm}
\addtolength{\textheight}{4cm}
\addtolength{\hoffset}{-1cm}
\addtolength{\textwidth}{2cm}

\def\figpathnew{figures/}
\newtheorem{theorem}{Theorem}%[section]
\newtheorem{lemma}[theorem]{Lemma}
\newtheorem{corollary}[theorem]{Corollary}       
\newtheorem{proposition}[theorem]{Proposition}
\newtheorem{definition}[theorem]{Definition}
\newtheorem{example}[theorem]{Example}

\newtheorem{remark}{Remark}
\newtheorem{algorithm}{Algorithm}

\DeclareMathOperator{\conv}{conv}
\DeclareMathOperator{\dist}{dist}
\DeclareMathOperator{\interior}{int}
\DeclareMathOperator{\ext}{ext}
\DeclareMathOperator{\sign}{sign}


\def\R{\mathbb{R}}
\def\N{\mathbb{N}}

\def\proof{{\noindent \it Proof. }} 
\def\whbox{\mbox{\vrule\vbox to 7pt{\hrule width 7pt\vss\hrule}\vrule}}
\def\qed{{\parfillskip = 0pt\nobreak\hskip 3em\penalty50\mbox{}\hfil
             \whbox\par}}

\begin{document}


\section{Outer convex approximation}\label{OuterConvexApproximation}
Let
\begin{equation}\label{def_X1}
X := \{x_1, x_2, \dots, x_n\} \subset \R^2.
\end{equation}
Assume without loss of generality that
\begin{equation}\label{def_X2}
\mbox{$x_1, x_2, \dots, x_n$ do not lie on the same straight line.}
\end{equation}

We write all vectors as row vectors, whose transposes are denoted by the superscript $T$, and use 
the upper indices to designate their components, e.g. $x = (x^1, x^2) \in \R^2$. For $x, x' \in \R^2$, denote 
\begin{equation*}
\begin{array}{lcl}
[x, x'] &:=& \{(1-\lambda) x + \lambda x' \mid \lambda\in [0, 1]\}, \\
(x, x') &:=& \{(1-\lambda) x + \lambda x' \mid \lambda\in (0, 1)\}.
\end{array}
\end{equation*}
Given $X$ satisfying (\ref{def_X1})--(\ref{def_X2}) and $\delta \geq 0$,
in this section we want to find an outer convex approximation of $X$, i.e.
\begin{equation}\label{def_calPouter-1}
\mbox{a convex polygon ${\cal P}^{\rm outer}$ satisfying $\conv X \subset {\cal P}^{\rm outer}$} 
\end{equation}
such that
\begin{equation}\label{def_calPouter-2}
\dist_{\rm H}(\conv X, {\cal P}^{\rm outer}) \leq \delta.
\end{equation}

Let us determine ${\cal P}^{\rm outer}$ by
\begin{equation}\label{def_calPouter0}
{\cal P}^{\rm outer} := \{x \in \R^2 \mid d\, x^T \leq \beta_d \ \mbox{ for all } d \in D\},
\end{equation}
where $D \subset \R^2$ denotes the set of maximum directions and $\beta_d \in \R$ denotes the threshold corresponding to the direction $d \in D$.
For given $D$, ${\cal P}^{\rm outer}$ is the best outer convex approximation polygon containing $X$ if
\begin{equation}\label{def_betad}
\beta_d := \max_{x \in X} d\, x^T \ \mbox{ for all } d \in D.
\end{equation}
Let $P$ be the vertex set of ${\cal P}^{\rm outer}$.

We start our outer convex approximation process with the smallest rectangle containing $X$ whose edges are parallel to the coordinate axes. According to (\ref{def_calPouter0})--(\ref{def_betad}), this rectangle ${\cal P}^{\rm outer}$ is determined by
\begin{equation}\label{def_D1}
D := \{(1, 0),\ (0, 1),\ (-1, 0),\ (0, -1)\}
\end{equation}
and
\begin{equation}\label{def_beta1}
\begin{array}{lcl}
\beta_{(1, 0)} &:=& \max\{x^1 \mid (x^1, x^2) \in X\}, \\
\beta_{(0, 1)} &:=& \max\{x^2 \mid (x^1, x^2) \in X\}, \\
\beta_{(-1, 0)} &:=& \max\{-x^1 \mid (x^1, x^2) \in X\}, \\
\beta_{(0, -1)} &:=& \max\{-x^2 \mid (x^1, x^2) \in X\}.
\end{array}
\end{equation}
By (\ref{def_X2}) we have
\begin{equation*}
\beta_{(-1, 0)} < \beta_{(1, 0)} \ \mbox{ and } \ \beta_{(0, -1)} < \beta_{(0, 1)}.
\end{equation*}
Therefore, ${\cal P}^{\rm outer}$ is a proper rectangle with four distinct vertices, whose vertex set is 
\begin{equation}\label{def_4r-2}
P := \{r_1,\ r_2,\ r_3,\ r_4\},
\end{equation}
where
\begin{equation}\label{def_4r-1}
\begin{array}{lcl}
r_1 &:=& (\beta_{(1, 0)}, \beta_{(0, 1)}), \\
r_2 &:=& (\beta_{(-1, 0)}, \beta_{(0, 1)}), \\
r_3 &:=& (\beta_{(-1, 0)}, \beta_{(0, -1)}), \\
r_4 &:=& (\beta_{(1, 0)}, \beta_{(0, -1)}).
\end{array}
\end{equation}

In the following approximation steps, the constructed polygon ${\cal P}^{\rm outer}$ is successively improved as follows.

For a vertex $p \in P$, let
\begin{equation}\label{def_p-p+}
\mbox{$p^- \in P$ and $p^+ \in P$ be the counterclockwise predecessor and successor of $p\in P$}
\end{equation}
and denote
\begin{equation}\label{def_d_p}
\begin{array}{lcl}
d_{p}^T &:=& \|p^+ - p^-\|^{-1}\, R \, (p^+ - p^-)^T, \\
\beta_{d_{p}} &:=& \max\{d_{p}\, x^T \mid x \in X\},
\end{array}
\end{equation}
where 
\begin{equation}\label{rotationmatrix}
R := \begin{pmatrix}
0 & 1 \\
-1 & 0
\end{pmatrix}
\end{equation}
is the clockwise rotation matrix through angle $\pi/2$.
As $R$ is a rotation matrix, we  have
\begin{equation}\label{def_d_p1}
\|d_{p}\| = \|p^+ - p^-\|^{-1}\, \|(p^+ - p^-) R^T\| = \|p^+ - p^-\|^{-1}\, \|p^+ - p^-\| = 1.
\end{equation}

There are two cases where we add the following linear constraint to the definition of ${\cal P}^{\rm outer}$ in (\ref{def_calPouter0}):
\begin{equation}\label{linearconstr}
d_p\, x^T \leq \beta_{d_p}.
\end{equation}

First, if
\begin{equation}\label{betaequal}
\beta_{d_p} = d_p\, p^+
\end{equation}
% (i.e., $\beta_{d_p} = d_p\, p^- = d_p\, p^+$), 
then the new constraint (\ref{linearconstr}) creates no new vertex but the new edge $[p^-, p^+]$ of ${\cal P}^{\rm outer}$.
Let $d_{[p^-, p]}$ and $d_{[p, p^+]}$ denote the two maximum directions from $D$ which define the two edges $[p^-, p]$ and $[ p, p^+]$ of ${\cal P}^{\rm outer}$. Then $d_{[p^-, p]}$ and $d_{[p, p^+]}$ become superfluous. Therefore, while adding $d_p$ to $D$ we have to eliminate $d_{[p^-, p]}$ and $d_{[p, p^+]}$ from $D$ and $p$ from $P$, i.e.
\begin{equation}\label{newDP2}
\begin{array}{lcl}
D &:=& (D \cup \{d_{p}\})\setminus \{d_{[p^-,p]}, d_{[p,p^+]}\}, \\
P &:=& P \setminus \{p\}.
\end{array}
\end{equation}

Second, if 
\begin{equation}\label{betagreater}
\beta_{d_p} > d_p\, p^+
\end{equation}
% (i.e., $\beta_{d_p} > d_p\, p^- = d_p\, p^+$) 
and
\begin{equation}\label{greaterdelta}
d_{p}\, p^T - \beta_{d_{p}} > \delta
\end{equation}
then the new constraint (\ref{linearconstr}) creates two new vertices of ${\cal P}^{\rm outer}$ named $\hat p^-$ and $\hat p^+$ that can be computed by
\begin{equation}\label{def_hatp}
\begin{array}{lcl}
\lambda_p &:=& (\beta_{d_p} - d_p\, p^{-T})/(d_p\, p^T - d_p\, p^{-T}) \in (0, 1), \\
\hat p^- &:=& (1 - \lambda_p)\, p^{-T} + \lambda_p\, p^T, \\
\hat p^+ &:=& (1 - \lambda_p)\, p^{+T} + \lambda_p\, p^T.
\end{array}
\end{equation}
Hence, we have to add $d_p$ to $D$ and replace $p \in P$ by $\hat p^-$ and $\hat p^+$, i.e.
\begin{equation}\label{newDP1}
\begin{array}{lcl}
D &:=& D \cup \{d_{p}\}, \\
P &:=&(P \setminus \{p\}) \cup \{\hat p^-, \hat p^+\}.
\end{array}
\end{equation}

The approximation procedure is described in the following algorithm, where $P_{\rm doubt}$ denotes the set of vertices that still need to be checked.

\begin{algorithm}\label{alg01}  \rm \ \\
\emph{Input:} The finite set $X \subset \R^2$ and the approximation parameter $\delta \geq 0$. \\
\emph{Output:} The outer convex approximation polygon ${\cal P}^{\rm outer}$ defined in (\ref{def_calPouter0}) by $D$ and $\beta_d$ for $d \in D$ and its vertex set $P$.
\begin{enumerate}[I.]
\item\label{stepIalg01} 
Determine $D$, $\beta_d$ for $d \in D$, and $P$ by (\ref{def_D1})--(\ref{def_4r-1}).

\item\label{stepIIalg01} Set $P_{\rm doubt} := P$.

\item\label{stepIIIalg01} 
Choose an arbitrary $p \in P_{\rm doubt}$ and determine $d_p$, $\beta_{d_p}$ by (\ref{def_p-p+})--(\ref{rotationmatrix}). \\
If (\ref{betaequal}) is true then change $D$, $P$ as in (\ref{newDP2}), update 
\begin{equation}\label{newPdoubt2}
P_{\rm doubt} := P_{\rm doubt} \setminus \{p, p^-, p^+\},
\end{equation}
and go to Step \ref{stepIValg01}.\\
If % (\ref{betagreater}) and 
(\ref{greaterdelta}) is true then change $D$, $P$ as in (\ref{newDP1}), update 
\begin{equation}\label{newPdoubt1}
P_{\rm doubt} := (P_{\rm doubt} \setminus \{p\}) \cup \{\hat p^-, \hat p^+\},
\end{equation}
and go to Step \ref{stepIValg01}.\\
Otherwise, 
\begin{equation}\label{newPdoubt3}
P_{\rm doubt} := P_{\rm doubt} \setminus \{p\}.
\end{equation}

\item\label{stepIValg01} 
If $P_{\rm doubt}$ is nonempty then go to Step \ref{stepIIIalg01}.

\item
Return the set of maximal directions $D$ and $\beta_d$ for $d \in D$, the vertex set $P$ of ${\cal P}^{\rm outer}$, and stop.
\end{enumerate}
\end{algorithm}

By (\ref{def_calPouter0})--(\ref{def_betad}), 
%\begin{equation}\label{every_edge}
%\mbox{
every edge $[p, p^+]$ of $\cal P$ contains at least one point of $X$.
%}
%\end{equation}
Therefore, (\ref{def_d_p}) implies
\begin{equation*}
\beta_{d_p} \geq d_p\, p^- = d_p\, p^+.
\end{equation*}
Hence, if (\ref{betaequal}) is false then (\ref{betagreater}) is automatically true.
This explains why (\ref{betagreater}) does not need to be checked together with (\ref{greaterdelta}) in Step \ref{stepIIIalg01}.

Algorithm \ref{alg01} can quickly provide an outer convex approximation polygon ${\cal P}^{\rm outer}$ with
\begin{equation}\label{smallerdelta}
d_{p}\, p^T - \beta_{d_{p}} \leq \delta \ \mbox{ for all } p \in P.
\end{equation}
This property is necessary but not sufficient for (\ref{def_calPouter-2}).
To ensure (\ref{def_calPouter-2}), in the next algorithm we also check if
\begin{equation}\label{checkdistance}
\begin{array}{lcl}
&& \|p - x\| > \delta \ \mbox{ for all } x \in X_p, \ \mbox{ where} \\
&& X_p := \{x \in X \mid d_{p}\, x^T = \beta_{d_{p}}\}.
\end{array}
\end{equation}

\begin{algorithm}\label{alg02}  \rm \ \\
\emph{Input:} The finite set $X \subset \R^2$ and the approximation parameter $\delta \geq 0$. \\
\emph{Output:} The outer convex approximation polygon ${\cal P}^{\rm outer}$ defined in (\ref{def_calPouter0}) by $D$ and $\beta_d$ for $d \in D$ and its vertex set $P$.
\begin{enumerate}[I.]
\item\label{stepIalg02} 
Determine $D$, $\beta_d$ for $d \in D$, and $P$ by (\ref{def_D1})--(\ref{def_4r-1}).

\item\label{stepIIalg02} Set $P_{\rm doubt} := P$.

\item\label{stepIIIalg02} 
Choose an arbitrary $p \in P_{\rm doubt}$ and determine $d_p$, $\beta_{d_p}$ by (\ref{def_p-p+})--(\ref{rotationmatrix}). \\
If (\ref{betaequal}) is true then change $D$, $P$ as in (\ref{newDP2}), update 
\begin{equation}\label{newPdoubt2b}
P_{\rm doubt} := P_{\rm doubt} \setminus \{p, p^-, p^+\},
\end{equation}
and go to Step \ref{stepIValg02}.\\
If (\ref{greaterdelta}) is true then change $D$, $P$ as in (\ref{newDP1}), update 
\begin{equation}\label{newPdoubt1b}
P_{\rm doubt} := (P_{\rm doubt} \setminus \{p\}) \cup \{\hat p^-, \hat p^+\},
\end{equation}
and go to Step \ref{stepIValg02}.\\
If (\ref{checkdistance}) is true then change $D$, $P$ as in (\ref{newDP1}), update $P_{\rm doubt}$ by (\ref{newPdoubt1b}), and go to Step \ref{stepIValg02}.\\
Otherwise, 
\begin{equation}\label{newPdoubt3b}
P_{\rm doubt} := P_{\rm doubt} \setminus \{p\}.
\end{equation}

\item\label{stepIValg02} 
If $P_{\rm doubt}$ is nonempty then go to Step \ref{stepIIIalg02}.

\item
Return the set of maximal directions $D$ and $\beta_d$ for $d \in D$, the vertex set $P$ of ${\cal P}^{\rm outer}$, and stop.
\end{enumerate}
\end{algorithm}


\begin{figure}[ht]
	\centering
	\includegraphics[width=7cm]{./Figures/ConvAppr-Figure01.eps}
	\caption{$X = \{x_1, x_2, \dots, x_n\}$ with $\conv X = \conv\{x_1, x_2, \dots, x_m\}$ and $m = 10$.}
	\label{Figure01}
\end{figure}



To illustrate the effect of Algorithms \ref{alg01}--\ref{alg02}, let us consider $X$ as a set of $n$ random points in the $16$-sided frame polygon ${\cal P}^\diamond$ shown in Figure \ref{Figure02}.

\begin{figure}[ht]
	\centering
    \includegraphics[width=12cm]{./Figures/Rotate-30.png}
	\caption{$16$-sided frame polygon ${\cal P}^\diamond$ of $n$ random points.}
	\label{Figure02}
\end{figure}

\medskip
Table \ref{table01} shows some experimental results, where
\begin{itemize}
\item $\#_{\rm Edges@ Alg.\, 1}$ is the average number of edges of the outer convex approximation polygon ${\cal P}^{\rm outer}$ returned by Algorithm \ref{alg01},
\item $\#_{\rm Edges@ Alg.\, 2}$ is the average number of edges of the outer convex approximation polygon ${\cal P}^{\rm outer}$ returned by Algorithm \ref{alg02},
\item $\#_{\rm Step\, III @ Alg.\, 1}$ is the average number of execution times of Step III in Algorithm \ref{alg01},
\item $\#_{\rm Step\, III @ Alg.\, 2}$ is the average number of execution times of Step III in Algorithm \ref{alg02}.
\end{itemize}

		\begin{table}[ht]
	\begin{center}\renewcommand{\arraystretch}{1.2}\small
		\setlength\tabcolsep{0.05cm}
		\begin{tabular}{|c|c||c|c|c|c|c|c|c|c|c|c|c|c|c|}
		%\hline
		\hline
		\multicolumn {2}{|c||}{\footnotesize $\#X=n$}  & ???& ???& ???& ???& ???& ???& ???& ???& ???& ???& ??? \\ 
		\hline		
		\hline
		\multirow{ 4}{*}{ $\delta = ???$}

        & $\#_{\rm Edges@ Alg.\, 1}$  &   ???& ???& ???& ???& ???& ???& ???& ???& ???& ???& ??? \\
        
        & $\#_{\rm Edges@ Alg.\, 2}$  &   ???& ???& ???& ???& ???& ???& ???& ???& ???& ???& ??? \\
		
		& $\#_{\rm Step\, III @ Alg.\, 1}$  &   ???& ???& ???& ???& ???& ???& ???& ???& ???& ???& ??? \\
		
		& $\#_{\rm Step\, III @ Alg.\, 2}$& ???& ???& ???& ???& ???& ???& ???& ???& ???& ???& ???   \\
		\hline
		\multirow{ 4}{*}{ $\delta = ???$}

        & $\#_{\rm Edges@ Alg.\, 1}$  &   ???& ???& ???& ???& ???& ???& ???& ???& ???& ???& ??? \\
        
        & $\#_{\rm Edges@ Alg.\, 2}$  &   ???& ???& ???& ???& ???& ???& ???& ???& ???& ???& ??? \\
		
		& $\#_{\rm Step\, III @ Alg.\, 1}$  &   ???& ???& ???& ???& ???& ???& ???& ???& ???& ???& ??? \\
		
		& $\#_{\rm Step\, III @ Alg.\, 2}$& ???& ???& ???& ???& ???& ???& ???& ???& ???& ???& ???   \\
		\hline
		\multirow{ 4}{*}{ $\delta = ???$}

        & $\#_{\rm Edges@ Alg.\, 1}$  &   ???& ???& ???& ???& ???& ???& ???& ???& ???& ???& ??? \\
        
        & $\#_{\rm Edges@ Alg.\, 2}$  &   ???& ???& ???& ???& ???& ???& ???& ???& ???& ???& ??? \\
		
		& $\#_{\rm Step\, III @ Alg.\, 1}$  &   ???& ???& ???& ???& ???& ???& ???& ???& ???& ???& ??? \\
		
		& $\#_{\rm Step\, III @ Alg.\, 2}$& ???& ???& ???& ???& ???& ???& ???& ???& ???& ???& ???   \\
		\hline
		\multirow{ 4}{*}{ $\delta = 0$}

        & $\#_{\rm Edges@ Alg.\, 1}$  &   ???& ???& ???& ???& ???& ???& ???& ???& ???& ???& ??? \\
        
        & $\#_{\rm Edges@ Alg.\, 2}$  &   ???& ???& ???& ???& ???& ???& ???& ???& ???& ???& ??? \\
		
		& $\#_{\rm Step\, III @ Alg.\, 1}$  &   ???& ???& ???& ???& ???& ???& ???& ???& ???& ???& ??? \\
		
		& $\#_{\rm Step\, III @ Alg.\, 2}$& ???& ???& ???& ???& ???& ???& ???& ???& ???& ???& ???   \\
		\hline
	\end{tabular}
		\caption{The average number of edges of the outer convex approximation polygon ${\cal P}^{\rm outer}$ returned by Algorithms \ref{alg01}--\ref{alg02} and the average number of execution times of their Step III when $X$ consists of $n$ random points in the $16$-sided frame polygon ${\cal P}^\diamond$ shown in Figure \ref{Figure02}.}
		\label{table01}
	\end{center}
\end{table} 	



\red{
\bigskip\noindent
{\bf REMARK (concerning Table \ref{table01}):} 
The $n$ random points presented in Table \ref{table01} must be generated in the rotated 16-sided frame polygon ${\cal P}^\diamond$ shown in Figure \ref{Figure02}.

\bigskip
}

If the point number $n$ of $X$ is very large then the computation of $\beta_{d_{p}}$ defined in (\ref{def_d_p}) is expensive, which can be significantly reduced as follows. Take four points $q_1$, $q_2$, $q_3$, and $q_4$ of $X$ that lie on the four edges of the starting rectangle ${\cal P}^{\rm outer}$, i.e.
\begin{equation}\label{def_4q-1}
q_1, q_2, q_3, q_4 \in X
\end{equation}
and
\begin{equation}\label{def_4q-2}
\begin{array}{lcl}
(1, 0)\, q_1^T &=& \beta_{(1, 0)}, \\
(0, 1)\, q_2^T &=& \beta_{(0, 1)}, \\
(-1, 0)\, q_3^T &=& \beta_{(-1, 0)}, \\
(0, -1)\, q_4^T &=& \beta_{(0, -1)}.
\end{array}
\end{equation}
For $j \in \{1, 2, 3, 4\}$ and $q_5 := q_1$, if $q_j \not= q_{j+1}$ then determine
\begin{equation}\label{def_4Xj}
X_j := \{x \in X \mid \bar d_{[q_j, q_{j+1}]}\, (x - q_j)^T \geq 0\}, 
\end{equation}
where
\begin{equation}\label{def_4dqj}
\bar d_{[q_j, q_{j+1}]}^{\, T} := \|q_{j+1} - q_j\|^{-1} R\, (q_{j+1} - q_j)^T.
\end{equation}
In the next algorithm, $X$ in (\ref{def_d_p}) and (\ref{checkdistance}) is replaced by $X_j$, i.e.
\begin{equation}\label{def_d_p-new}
\begin{array}{lcl}
d_{p}^T &:=& \|p^+ - p^-\|^{-1} R \, (p^+ - p^-)^T, \\
\beta_{d_{p}} &:=& \max\{d_{p}\, x^T \mid x \in X_j\}, 
\end{array}
\end{equation}
and
\begin{equation}\label{checkdistance+}
\begin{array}{lcl}
&& \|p - x\| > \delta \ \mbox{ for all } x \in X_p, \ \mbox{ where} \\
&& X_p := \{x \in X_j \mid d_{p}\, x^T = \beta_{d_{p}}\},
\end{array}
\end{equation}
for $j \in \{1, 2, 3, 4\}$.

\begin{algorithm}\label{alg03}  \rm \ \\
\emph{Input:} The finite set $X \subset \R^2$ and the approximation parameter $\delta \geq 0$. \\
\emph{Output:} The outer convex approximation polygon ${\cal P}^{\rm outer}$ defined in (\ref{def_calPouter0}) by $D$ and $\beta_d$ for $d \in D$ and its vertex set $P$.
\begin{enumerate}[I.]
\item\label{stepIalg03} 
Determine $D$, $\beta_d$ for $d \in D$, $r_j$ for $j \in \{1, 2, 3, 4\}$, and $P$ by (\ref{def_D1})--(\ref{def_4r-1}). \\
Set $j := 1$ and $q_5 := q_1$. 

\item\label{stepIIalg03} 
If $q_j = q_{j+1}$ then go to Step \ref{stepValg03}.\\
Otherwise, determine $X_j$ by (\ref{def_4q-1})--(\ref{def_4Xj}) and set $P_{\rm doubt} := \{r_j\}$.

\item\label{stepIIIalg03} 
Choose an arbitrary $p \in P_{\rm doubt}$ and determine $d_p$, $\beta_{d_p}$ by (\ref{def_p-p+}) and (\ref{def_d_p-new}). \\
If
\begin{equation*}%\label{betaequal+}
\beta_{d_p} = d_p\, p^-
\end{equation*}
then change $D$ and $P$ as in (\ref{newDP2}), update 
\begin{equation*}\label{newPdoubt5+}
P_{\rm doubt} := P_{\rm doubt} \setminus \{p, p^-, p^+\}
\end{equation*}
and go to Step \ref{stepIValg03}.\\
If 
\begin{equation*}%\label{greaterdelta+}
d_{p}\, p^T - \beta_{d_{p}} > \delta
\end{equation*}
then change $D$ and $P$ as in (\ref{newDP1}), update 
\begin{equation}\label{newPdoubt4+}
P_{\rm doubt} := (P_{\rm doubt} \setminus \{p\}) \cup \{\hat p^-, \hat p^+\}
\end{equation}
and go to Step \ref{stepIValg03}.\\
If (\ref{checkdistance+}) is true then change $D$, $P$ as in (\ref{newDP1}), update $P_{\rm doubt}$ by (\ref{newPdoubt4+}), and go to Step \ref{stepIValg03}.\\
Otherwise,
\begin{equation*}%\label{newPdoubt6+}
P_{\rm doubt} := P_{\rm doubt} \setminus \{p\}.
\end{equation*}

\item\label{stepIValg03} 
If $P_{\rm doubt}$ is nonempty then go to Step \ref{stepIIIalg03}.

\item\label{stepValg03} 
If $j < 4$ then $j := j+1$ and go to Step \ref{stepIIalg03}.

\item\label{stepVIalg03} 
Return the set of maximal directions $D$ and $\beta_d$ for $d \in D$, the vertex set $P$ of ${\cal P}^{\rm outer}$, and stop.
\end{enumerate}
\end{algorithm}



\medskip
Table \ref{table02} and Figure \ref{Figure03} show some experimental results on the running time of Algorithms \ref{alg02}--\ref{alg03} when $X$ consists of $n$ random points in the $16$-sided frame polygon ${\cal P}^\diamond$ shown in Figure \ref{Figure02}, where
\begin{itemize}
\item $T_{\rm Alg.\, 2}$ is the average running time of Algorithm \ref{alg02},
\item $T_{\rm Alg.\, 3}$ is the average running time of Algorithm \ref{alg03}.
\end{itemize}

		\begin{table}[ht]
	\begin{center}\renewcommand{\arraystretch}{1.2}\small
		\setlength\tabcolsep{0.05cm}
		\begin{tabular}{|c|c||c|c|c|c|c|c|c|c|c|c|c|c|c|}
		%\hline
		\hline
		\multicolumn {2}{|c||}{\footnotesize $\#X=n$}  & ???& ???& ???& ???& ???& ???& ???& ???& ???& ???& ??? \\ 
		\hline		
		\hline
		\multirow{ 4}{*}{ $\delta = ???$}

        & $T_{\rm Alg.\, 2}$  &   ???& ???& ???& ???& ???& ???& ???& ???& ???& ???& ??? \\
        
        & $T_{\rm Alg.\, 3}$  &   ???& ???& ???& ???& ???& ???& ???& ???& ???& ???& ??? \\
		
		& $T_{\rm Alg.\, 2}/n$  &   ???& ???& ???& ???& ???& ???& ???& ???& ???& ???& ??? \\
		
		& $T_{\rm Alg.\, 3}/n$& ???& ???& ???& ???& ???& ???& ???& ???& ???& ???& ???   \\
		\hline
		\multirow{ 4}{*}{ $\delta = ???$}

        & $T_{\rm Alg.\, 2}$  &   ???& ???& ???& ???& ???& ???& ???& ???& ???& ???& ??? \\
        
        & $T_{\rm Alg.\, 3}$  &   ???& ???& ???& ???& ???& ???& ???& ???& ???& ???& ??? \\
		
		& $T_{\rm Alg.\, 2}/n$  &   ???& ???& ???& ???& ???& ???& ???& ???& ???& ???& ??? \\
		
		& $T_{\rm Alg.\, 3}/n$& ???& ???& ???& ???& ???& ???& ???& ???& ???& ???& ???   \\
		\hline
		\multirow{ 4}{*}{ $\delta = ???$}

        & $T_{\rm Alg.\, 2}$  &   ???& ???& ???& ???& ???& ???& ???& ???& ???& ???& ??? \\
        
        & $T_{\rm Alg.\, 3}$  &   ???& ???& ???& ???& ???& ???& ???& ???& ???& ???& ??? \\
		
		& $T_{\rm Alg.\, 2}/n$  &   ???& ???& ???& ???& ???& ???& ???& ???& ???& ???& ??? \\
		
		& $T_{\rm Alg.\, 3}/n$& ???& ???& ???& ???& ???& ???& ???& ???& ???& ???& ???   \\
		\hline
		\multirow{ 4}{*}{ $\delta = 0$}

        & $T_{\rm Alg.\, 2}$  &   ???& ???& ???& ???& ???& ???& ???& ???& ???& ???& ??? \\
        
        & $T_{\rm Alg.\, 3}$  &   ???& ???& ???& ???& ???& ???& ???& ???& ???& ???& ??? \\
		
		& $T_{\rm Alg.\, 2}/n$  &   ???& ???& ???& ???& ???& ???& ???& ???& ???& ???& ??? \\
		
		& $T_{\rm Alg.\, 3}/n$& ???& ???& ???& ???& ???& ???& ???& ???& ???& ???& ???   \\
		\hline
	\end{tabular}
		\caption{The average running time $T_{\rm Alg.\, 2}$ and $T_{\rm Alg.\, 3}$ of Algorithm \ref{alg02} and Algorithm \ref{alg03}
when $X$ consists of $n$ random points in the $16$-sided frame polygon ${\cal P}^\diamond$ shown in Figure \ref{Figure02}.}
		\label{table02}
	\end{center}
\end{table} 	

\begin{figure}[ht]
	\centering
	???
%\includegraphics[width=8cm]{./Figures/???}
	\caption{The ratio between the running time of Algorithms \ref{alg02}--\ref{alg03} and the point number $n$ of $X$.}
	\label{Figure03}
\end{figure}


\red{
\bigskip\noindent
{\bf REMARK (concerning Table \ref{table02} and Figure \ref{Figure03}):} 

 - The $n$ random points presented in Table \ref{table02} must be generated in the rotated 16-sided frame polygon ${\cal P}^\diamond$ shown in Figure \ref{Figure02}.
 
 - Only two ratios $T_{\rm Alg.\, 2}/n$ and $T_{\rm Alg.\, 3}/n$ are presented in Figure \ref{Figure03}.

\bigskip
} 




\section{Inner convex approximation}\label{InnerConvexApproximation}

Given $X$ satisfying (\ref{def_X1})--(\ref{def_X2}) and $\delta \geq 0$,
in this section we want to find an inner convex approximation ${\cal P}^{\rm inner}$  of $X$, i.e.
\begin{equation}\label{def_calPinner-1}
{\cal P}^{\rm inner} := \conv X', \ \mbox{ where } X' \subset X,
\end{equation}
such that
\begin{equation}\label{def_calPinner-2}
\dist_{\rm H}(\conv X, {\cal P}^{\rm inner}) \leq \delta.
\end{equation}

Let us describe ${\cal P}^{\rm inner}$ by the set $E$ of its directed edges $[p, p^+]$, where $p^+$ is a counterclockwise successor of $p$, i.e.
\begin{equation}\label{def_E}
E := \{[p, p^+] \mid p, p^+ \in X', \mbox{ $[p, p^+]$ is an edge of ${\cal P}^{\rm inner}$}\}.
\end{equation}


We start our inner convex approximation process with the quadrilateral $\bar q_1 \bar q_2 \bar q_3 \bar q_4$, i.e.
\begin{equation}\label{def_X'E}
\begin{array}{lcl}
X' &:=& \{\bar q_1, \bar q_2, \bar q_3, \bar q_4\}, \\
E &:=& \{[\bar q_1, \bar q_2], \, [\bar q_2, \bar q_3], \, [\bar q_3, \bar q_4], \, [\bar q_4, \bar q_1]\},
\end{array}
\end{equation}
where $\bar q_1$, $\bar q_2$, $\bar q_3$, and $\bar q_4$ are uniquely defined by
\begin{equation}\label{def_4barq1}
\begin{array}{lcl}
x^1_{\rm min} &:=& \min \{x^1 \mid (x^1, x^2) \in X\}, \\
x^1_{\rm max} &:=& \max \{x^1 \mid (x^1, x^2) \in X\}, \\
x^2_{\rm min} &:=& \min \{x^2 \mid (x^1, x^2) \in X\}, \\
x^2_{\rm max} &:=& \max \{x^2 \mid (x^1, x^2) \in X\}
\end{array}
\end{equation}
and 
\begin{equation}\label{def_4barq2}
\begin{array}{lcl}
X^1_{\rm min} &:=& \{(x^1, x^2) \in X \mid x^1 = x^1_{\rm min}\}, \\
X^1_{\rm max} &:=& \{(x^1, x^2) \in X \mid x^1 = x^1_{\rm max}\}, \\
X^2_{\rm min} &:=& \{(x^1, x^2) \in X \mid x^2 = x^2_{\rm min}\}, \\
X^2_{\rm max} &:=& \{(x^1, x^2) \in X \mid x^2 = x^2_{\rm max}\}
\end{array}
\end{equation}
and 
\begin{equation}\label{def_4barq3}
\begin{array}{lcl}
&& \bar q_1 = (\bar q_1^1, \bar q_1^2) \in X^1_{\rm max} \mbox{ satisfying } \bar q_1^2 = \max\{x^2 \mid (x^1, x^2) \in X^1_{\rm max}\}, \\
&& \bar q_2 = (\bar q_2^1, \bar q_2^2) \in X^2_{\rm max} \mbox{ satisfying } \bar q_2^1 = \min\{x^1 \mid (x^1, x^2) \in X^2_{\rm max}\}, \\
&& \bar q_3 = (\bar q_3^1, \bar q_3^2) \in X^1_{\rm min} \mbox{ satisfying } \bar q_3^2 = \min\{x^2 \mid (x^1, x^2) \in X^1_{\rm min}\}, \\
&& \bar q_4 = (\bar q_4^1, \bar q_4^2) \in X^2_{\rm min} \mbox{ satisfying } \bar q_4^1 = \max\{x^1 \mid (x^1, x^2) \in X^2_{\rm min}\}.
\end{array}
\end{equation}
Note that two points of $\bar q_1$, $\bar q_2$, $\bar q_3$, and $\bar q_4$ can coincide, but (\ref{def_X2}) implies that at least three of them are different.

In the following approximation steps, the constructed polygon ${\cal P}^{\rm inner}$ is successively
improved as follows.
For any $[p, p^+] \in E$ ($p \not= p^+$), define
\begin{equation}\label{def_d&X_d}
\begin{array}{lcl}
\bar d_{[p, p^+]}^{\, T} &:=& \|p^+ - p\|^{-1} R \, (p^+ - p)^T, \\
X_{[p, p^+]} &:=& \{x \in X \mid \bar d_{[p, p^+]}\, x^T > \bar d_{[p, p^+]}\, p^T \},
\end{array}
\end{equation}
where
\begin{equation*}%\label{rotationmatrix}
R := \begin{pmatrix}
0 & 1 \\
-1 & 0
\end{pmatrix}.
\end{equation*}

If $X_{[p, p^+]} \not= \emptyset$ then determine
\begin{equation}\label{beta1}
\begin{array}{lcl}
\beta_{[p, p^+]} &:=& \max \{\bar d_{[p, p^+]}\, x^T \mid x \in X_{[p, p^+]}\}, \\
B_{[p, p^+]} &:=& \{x \in X_{[p, p^+]} \mid \bar d_{[p, p^+]}\, x^T = \beta_{[p, p^+]}\}.
\end{array}
\end{equation}
If 
\begin{equation}\label{beta2}
\beta_{[p, p^+]} - \bar d_{[p, p^+]}\, p^T \leq \delta
\end{equation}
then we no longer have to extend ${\cal P}^{\rm inner }$ in the direction $\bar d_{[p, p^+]}$.

Otherwise, if 
\begin{equation}\label{beta3}
\beta_{[p, p^+]} - \bar d_{[p, p^+]}\, p^T > \delta
\end{equation}
then take   
\begin{equation}\label{beta4}
\hat p \in B_{[p, p^+]} \hbox{ satisfying } \|\hat p - p\| = \max\{\|x - p\| \mid x \in B_{[p, p^+]}\},
\end{equation}
update $X'$ and $B$ by
\begin{equation}\label{beta5}
\begin{array}{lcl}
X' &:=& X' \cup \{\hat p\}, \\
E &:=& E \cup \{[p, \hat p], \, [\hat p, p^+]\},
\end{array}
\end{equation}
and determine
\begin{equation}\label{beta6}
\begin{array}{lcl}
X_{[p, \hat p]} &:=& \{x \in X_{[p, p^+]} \mid \bar d_{[p, \hat p]}\, x^T > \bar d_{[p, \hat p]}\, p^T \}, \\
X_{[\hat p, p^+]} &:=& \{x \in X_{[p, p^+]} \mid \bar d_{[\hat p, p^+]}\, x^T > \bar d_{[\hat p, p^T]}\, {\hat p}^T \}.
\end{array}
\end{equation}
Note that we consider only $x \in X_{[p, p^+]}$ in this definition of $X_{[p, \hat p]}$ and $X_{[\hat p, p^+]}$, but all $x \in X$ in the definition (\ref{def_d&X_d}) of $X_{[p, p^+]}$.

The approximation procedure is presented in the following algorithm, where $E_{\rm doubt}$ denotes the set of edges that still need to be checked.

\begin{algorithm}\label{alg04}  \rm \ \\
\emph{Input:} The finite set $X \subset \R^2$ and the approximation parameter $\delta \geq 0$. \\
\emph{Output:} The inner convex approximation ${\cal P}^{\rm inner}$ described by $X'$ and $E$.
\begin{enumerate}[I.]
\item\label{stepIalg04} 
Determine $X'$ and $E$ by (\ref{def_X'E})--(\ref{def_4barq3}).\\
For all $[p, p^+] \in E$ determine $d_{[p, p^+]}$ and $X_{[p, p^+]}$ by (\ref{def_d&X_d}).\\
Set $E_{\rm doubt} := E$.

\item\label{stepIIalg04} 
Choose an arbitrary $[p, p^+] \in E_{\rm doubt}$. \\
If $X_{[p, p^+]} = \emptyset$ then set
$E_{\rm doubt} := E_{\rm doubt} \setminus \{[p, p^+]\}$
and go to Step \ref{stepIIIalg04}.\\
Determine $\beta_{[p, p^+]}$ and $B_{[p, p^+]}$ by (\ref{beta1}). \\
If (\ref{beta2}) is true then set
$E_{\rm doubt} := E_{\rm doubt} \setminus \{[p, p^+]\}$
and go to Step \ref{stepIIIalg04}.\\
Otherwise, take $\hat p$ defined by (\ref{beta4}),
update $X'$ and $E$ by (\ref{beta5}), determine $X_{[p, \hat p]}$ and $X_{[\hat p, p^+]}$ by (\ref{beta6}),
and set
\begin{equation*}%\label{newBdoubt3}
E_{\rm doubt} := (E_{\rm doubt} \setminus \{[p, p^+])\}) \cup \{[p, \hat p], \, [\hat p, p^+]\}.
\end{equation*}

\item\label{stepIIIalg04} 
If $E_{\rm doubt}$ is nonempty then go to Step \ref{stepIIalg04}.

\item
Return $X'$, $E$, and stop.
\end{enumerate}
\end{algorithm}


\medskip
Table \ref{table03} shows some experimental results, where
\begin{itemize}
\item $\#_{\rm Edges@ Alg.\, 4}$ is the average number of edges of the inner convex approximation polygon ${\cal P}^{\rm inner}$ returned by Algorithm \ref{alg04},
\item $\#_{\rm Step\, II @ Alg.\, 4}$ is the average number of execution times of Step \ref{stepIIalg04} in Algorithm \ref{alg04}.
\end{itemize}

		\begin{table}[ht]
	\begin{center}\renewcommand{\arraystretch}{1.2}\small
		\setlength\tabcolsep{0.05cm}
		\begin{tabular}{|c|c||c|c|c|c|c|c|c|c|c|c|c|c|c|}
		%\hline
		\hline
		\multicolumn {2}{|c||}{\footnotesize $\#X=n$}  & ???& ???& ???& ???& ???& ???& ???& ???& ???& ???& ??? \\ 
		\hline		
		\hline
		\multirow{2}{*}{ $\delta = ???$}

        & $\#_{\rm Edges@ Alg.\, 4}$  &   ???& ???& ???& ???& ???& ???& ???& ???& ???& ???& ??? \\
        		
		& $\#_{\rm Step\, II @ Alg.\, 4}$& ???& ???& ???& ???& ???& ???& ???& ???& ???& ???& ???   \\
		\hline
		\multirow{2}{*}{ $\delta = ???$}

        & $\#_{\rm Edges@ Alg.\, 4}$  &   ???& ???& ???& ???& ???& ???& ???& ???& ???& ???& ??? \\
        		
		& $\#_{\rm Step\, II @ Alg.\, 4}$& ???& ???& ???& ???& ???& ???& ???& ???& ???& ???& ???   \\
		\hline
		\multirow{2}{*}{ $\delta = ???$}

        & $\#_{\rm Edges@ Alg.\, 4}$  &   ???& ???& ???& ???& ???& ???& ???& ???& ???& ???& ??? \\
        		
		& $\#_{\rm Step\, II @ Alg.\, 4}$& ???& ???& ???& ???& ???& ???& ???& ???& ???& ???& ???   \\
		\hline
		\multirow{2}{*}{ $\delta = 0$}

        & $\#_{\rm Edges@ Alg.\, 4}$  &   ???& ???& ???& ???& ???& ???& ???& ???& ???& ???& ??? \\
        		
		& $\#_{\rm Step\, II @ Alg.\, 4}$& ???& ???& ???& ???& ???& ???& ???& ???& ???& ???& ???   \\
		\hline
	\end{tabular}
		\caption{The average number of edges of the inner convex approximation polygon ${\cal P}^{\rm inner}$ returned by Algorithm \ref{alg04} and the average number of execution times of its Step II when $X$ consists of $n$ random points in the $16$-sided frame polygon ${\cal P}^\diamond$ shown in Figure \ref{Figure02}.}
		\label{table03}
	\end{center}
\end{table} 	


\red{
\bigskip\noindent
{\bf REMARK (concerning Table \ref{table03}):} 
The $n$ random points presented in Table \ref{table03} must be generated in the rotated 16-sided frame polygon ${\cal P}^\diamond$ shown in Figure \ref{Figure02}.

\bigskip
}

\medskip
Table \ref{table04} and Figure \ref{Figure04} show some experimental results on the running time of Algorithm \ref{alg04} when $X$ consists of $n$ random points in the $16$-sided frame polygon ${\cal P}^\diamond$ shown in Figure \ref{Figure02}, where $T_{\rm Alg.\, 4}$ is the average running time of Algorithm \ref{alg04}.

		\begin{table}[ht]
	\begin{center}\renewcommand{\arraystretch}{1.2}\small
		\setlength\tabcolsep{0.05cm}
		\begin{tabular}{|c|c||c|c|c|c|c|c|c|c|c|c|c|c|c|}
		%\hline
		\hline
		\multicolumn {2}{|c||}{\footnotesize $\#X=n$}  & ???& ???& ???& ???& ???& ???& ???& ???& ???& ???& ??? \\ 
		\hline		
		\hline
		\multirow{2}{*}{ $\delta = ???$}

        & $T_{\rm Alg.\, 4}$  &   ???& ???& ???& ???& ???& ???& ???& ???& ???& ???& ??? \\
        		
		& $T_{\rm Alg.\, 4}/n$& ???& ???& ???& ???& ???& ???& ???& ???& ???& ???& ???   \\
		\hline
		\multirow{2}{*}{ $\delta = ???$}

        & $T_{\rm Alg.\, 4}$  &   ???& ???& ???& ???& ???& ???& ???& ???& ???& ???& ??? \\
        		
		& $T_{\rm Alg.\, 4}/n$& ???& ???& ???& ???& ???& ???& ???& ???& ???& ???& ???   \\
		\hline
		\multirow{2}{*}{ $\delta = ???$}

        & $T_{\rm Alg.\, 4}$  &   ???& ???& ???& ???& ???& ???& ???& ???& ???& ???& ??? \\
        		
		& $T_{\rm Alg.\, 4}/n$& ???& ???& ???& ???& ???& ???& ???& ???& ???& ???& ???   \\
		\hline
		\multirow{2}{*}{ $\delta = ???$}

        & $T_{\rm Alg.\, 4}$  &   ???& ???& ???& ???& ???& ???& ???& ???& ???& ???& ??? \\
        		
		& $T_{\rm Alg.\, 4}/n$& ???& ???& ???& ???& ???& ???& ???& ???& ???& ???& ???   \\
		\hline
	\end{tabular}
		\caption{The average running time $T_{\rm Alg.\, 4}$  Algorithm \ref{alg04} 
when $X$ consists of $n$ random points in the $16$-sided frame polygon ${\cal P}^\diamond$ shown in Figure \ref{Figure02}.}
		\label{table04}
	\end{center}
\end{table} 	

\begin{figure}[ht]
	\centering
	???
%\includegraphics[width=8cm]{./Figures/???}
	\caption{The ratio between the running time of Algorithm \ref{alg04} the point number $n$ of $X$.}
	\label{Figure04}
\end{figure}


\red{
\bigskip\noindent
{\bf REMARK (concerning Table \ref{table04} and Figure \ref{Figure04}):} 

 - The $n$ random points presented in Table \ref{table04} must be generated in the rotated 16-sided frame polygon ${\cal P}^\diamond$ shown in Figure \ref{Figure02}.
 
 - Only the ratio $T_{\rm Alg.\, 4}/n$ is presented in Figure \ref{Figure04}.

\bigskip
} 


\end{document}

